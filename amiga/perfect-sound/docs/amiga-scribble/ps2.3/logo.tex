\hsize=4in\vsize=6.9in\hoffset=-.25in\voffset=-.25in
\font\rm=optr10
\font\bf=optb10
\font\title=optb14
\font\sl=optsl10
\parindent=0pt\parskip=\smallskipamount
\let\tenrm=\rm
\centerline{\title Perfect Sound for the Amiga}
\medskip\hrule height 1.5pt\bigskip
\rm
\bigskip
{\obeylines
\centerline{\sl SunRize Industries}
\centerline{\sl P.O. Box 1453}
\centerline{\sl College Station, Texas 77841}
\centerline{\sl (409) 846-1311}
}
\bigskip
\bigskip
\bigskip
\bigskip
{\parshape=15 1.79in 0.65in 1.62in  0.98in  1.46in 1.31in 1.29in 1.64in
              1.13in 1.97in 0.965in 2.135in 0.8in  2.3in  1.9in  0.9in
              1.85in 0.85in 2.0in   0.8in   2.0in  0.8in  1.95in 0.85in
              1.9in  0.5in  1.75in  0.7in   1.5in  1.0in
\noindent This manual and the {\sl Perfect~Sound\/} software are
copyrighted 1986 by Anthony J. Wood. The {\sl Perfect~Sound\/}
editor executable module may be copied and distributed
for non--commercial use.  If you use the
editor for the purpose of profit, or in a
government institution (such as schools), the
program must be purchased. Contact your dealer or
the address above.
The {\sl Perfect Sound\/} audio digitizer is guaranteed for
90 days from the date of purchase.
If the hardware fails after this time period, return
the product to the address above for a repair estimate.}
\vfill\eject
\null\vfill\eject

{\title Files on Your Disk}
\smallskip\hrule height 1.5pt\bigskip
Important: Before using any of the programs on the disk
you received, make a backup and use the copy, not the
original.
\bigskip
{\obeylines
{\sl PERFECT SOUND}
      The Sound Editor Program
\smallskip
{\sl MONITOR}
      Monitors the digitizer using interrupts
\smallskip
{\sl FORMAT\_IFF}
      Text file with technical specs for the IFF format
\smallskip
{\sl SED}
      A directory containing the source code for {\sl Perfect Sound}
\smallskip
{\sl SOUNDS}
      A directory that contains digitized sounds in IFF format
}
\medskip
The disk you received does not contain Workbench.
If you have a one drive system, you may wish to copy {\sl Perfect~Sound\/}
on to a Workbench disk.
This is not absolutely necessary
since {\sl Perfect~Sound\/} can be loaded by booting with a Workbench
disk as usual, then removing Workbench and inserting the
the {\sl Perfect~Sound\/} disk.
The {\sl Perfect~Sound\/} disk icon will appear, and you can
double click it as usual.
\smallskip
If you wish to copy {\sl Perfect~Sound\/} on to a Workbench disk, two
methods can be used.
The first and simplest method
is as follows:
\smallskip
Boot your system as normal (insert Kickstart then Workbench).
Double click the Workbench icon to open the Workbench window.
Now, remove the Workbench disk, and insert the {\sl Perfect~Sound\/}
disk. Double click the {\sl Perfect~Sound\/} icon to open its window.
Once both the {\sl Perfect~Sound\/} and Workbench windows are open,
simply drag the recorder icon across to the Workbench window.
\smallskip
This method allows you to copy the {\sl Perfect~Sound\/} executable
program only, and does not copy the library of sounds.
To copy the recorded sounds and source files on to a
Workbench disk, you will need to use the CLI.
From the CLI:
\medskip
{\obeylines
(insert the {\sl Perfect~Sound\/} disk)
type: {\bf COPY ``PSOUND'' to RAM:}
\medskip
(now insert a Workbench disk)
type: {\bf COPY ``RAM:PSOUND'' to DF0:}

}
\smallskip
You can use this method to copy every file. The Workbench method
will only copy the {\sl Perfect Sound\/} executable module since
it is the only file that has an icon.
\vfill\eject

{\title Welcome}
\smallskip\hrule height 1.5pt
\bigskip
Thank you for choosing the {\sl SunRize~Perfect~Sound\/} digitizer.
If you have any questions, feel free to call
(409) 846-1311 for technical support.  Our normal hours are
from 9 a.m. to 6 p.m. Central time.
\smallskip
Using our sound digitizer adds a new dimension to your Amiga.
Sound can be added to {\sl Deluxe~Video\/} or your own programs.
If you have a music editor that uses IFF files, such as
{\sl Instant~Music\/}, you can create your own instruments.
{\sl Perfect~Sound\/}
comes with a complete editor that allows you
to modify sounds you record.  You can delete parts of a sound,
alter the order of a sound, change playback speeds, and even
see graphs of a sound.
\smallskip
Please do not to forget to send in your registration
-- {\sl Perfect Sound\/}
is constantly being upgraded.
\vfill\eject

{\title What is digitized sound?}
\smallskip\hrule height 1.5pt
\bigskip
The {\sl Perfect~Sound\/} audio digitizer converts an audio input
into a series of numbers that represent the input
sound. The input waveform is sampled 5,000 to 25,000 times
a second. Each {\sl sample} is a number from 0 to 255 that
indicates the amplitude of the input wave at the
instant in time when the sample was taken.
The higher the sampling rate you use, the better
the sound will be represented -- however, higher
sampling rates use more memory.  Digitized sound
uses a lot of memory. For example, two seconds of
reasonable stereo sound will easily take 50,000 bytes.
\smallskip
For a more complete discussion on digitized sound, see
the Amiga hardware reference manual.
\vfill\eject

{\title Installation}
\smallskip\hrule height 1.5pt
\bigskip
The {\sl Perfect~Sound\/}
digitizer plugs into the printer/parallel port
in the back of your Amiga. The two jacks on the side
of the digitizer are for connecting an
audio source. The bottom one is for the left channel, and the top one
is for the right channel. They do not both have to be hooked
up. If you just want to record a mono sound, connect it to
the bottom input. This is the left channel.
\vfill\eject

{\title Setting your digitizer}
\smallskip\hrule height 1.5pt
\bigskip
The two knobs on the front of your digitizer control
the gain of an input amplifier. The output from
this amplifier is what is digitized by an
analog to digital converter.
These knobs will have to be adjusted for different audio
sources, or even for different sounds on the
same audio source if you desire high quality
results.  For example, if you are digitizing a whisper
you may want to increase the gain of the input
amplifier.  This will allow you to get a range of
values from 0 to 255, thus using the full
precision of the A/D converter. This will result in
clearer sound. When you play the sound
back, it will be louder. However, just lower the volume
to get the whisper back.
\smallskip
To adjust these knobs for a source there are two methods that
can be used. The easiest is to select the {\sl "Monitor"\/} menu
under the {\sl digitize} menu. With an audio source connected,
adjust the knob at the front
of the digitizer to increase the volume. The idea is to
get the sound output as loud as possible without
introducing distortion.  Distortion will occur when
the gain of the input amplifier causes the signal levels
to pass out of the range of the A/D converter.
\smallskip
The second method is to pick the {\sl "Monitor Signal Levels"\/}
submenu option.  This will display actual numbers that
the A/D converter is producing.  In theory, you would want
to get the maximum value so it hits 255 only occasionally,
and the minimum value to hit 0 just as rarely. In this
way you are using the full range of 0 to 255, but
not going out of this range, which will result in distortion.
However, in practice, this method does not work too well,
and a slightly higher gain is usually better.
Try a combination of both methods until
you get a something that sounds good.
\smallskip
An easy way to check your input gain is to record a sample, and
then graph it. If you see a lot of flat peaks, your gain is probably
too high. See the
{\sl ``Graph~current~range''\/} option
under the {\sl ``Special''\/} section.
\smallskip
If you are recording in stereo, you need to
have the input gain of both channels identical. To
help you with this, the monitor signal levels submenu
displays the difference between the right and left
channels. Although it is impossible to get this
number to stay at zero (since the left and right inputs are
different), the smaller it is the better.
\vfill\eject

{\title Audio Sources}
\smallskip\hrule height 1.5pt
\bigskip
Almost any audio source can be used with {\sl Perfect~Sound\/}
-- stereos and VCRs are perhaps the most common.
Connect these two devices to the digitizer with a patch cord
(a shielded cable with two male RCA jacks on each end.)
{\sl Perfect~Sound\/} can also be connected to speaker jacks on
the back of radios or cassette players.
These sources, however, might not digitize as well as a line
level output, as in a VCR audio out.
You may need to purchase a mini--plug to RCA adapter
for some audio sources.  You can find these adapters
at most electronic stores.
Note that a microphone can not be plugged directly into the
digitizer.  If you wish to digitize from a microphone, you
need to go through a pre-amplifier. A small mini-amp
will work. Other possibilities are
small battery equalizers, or even your stereo.  Any device that
takes a mic input and a line or external speaker output
will work.
\vfill\eject

{\title Basic Sound Editor Functions}
\smallskip\hrule height 1.5pt
\bigskip
Once {\sl Perfect~Sound\/}
is loaded, you will notice five gadgets -- three sliders
and two boxes.
The three sliders are marked
{\sl start, end\/}
and
{\sl pos'n\/}
(position).  These adjust your start of range marker, end of
range marker, and insert position.  The sample that these
markers are altering is highlighted at the top of your screen.
While you are changing these markers, clicking the
right
mouse button will play the current sample up to the marker.
By using the start and end markers, you define a range
that editing commands work on. For example, you can
define a range, then delete that range by selecting
the
{\sl "Delete Marked Range"\/}
option in the
{\sl "Edit"\/}
menu.
On the right of each slider are two arrows. These are used to give
you precise control over the position of the slider. By clicking
these arrows, you can increase or decrease a value by one.  These
are particularly useful for editing instruments.
\smallskip
Two other gadgets are in the lower right corner of your
status area.  By clicking these gadgets, you can
play the current marked range, or the entire current sample.
\smallskip
When you load or record a sample, it will appear in the
upper half of your display. Up to 15 samples can
be in memory at the same time.  The current sample (the
sample that the start and end sliders adjust) will
be in inverse video. To choose a new current sample, you
simply click its name with the mouse.  Double clicking
a name will play the sample. To try this out, go to the
{\sl "File"\/}
menu and select
{\sl "Load"\/}
Change the directory to
{\bf sounds}
and load a few.
\smallskip
When you have loaded an instrument file, such as from
{\sl Instant~Music\/},
you can play the different octaves by using the function
keys. Each key, starting with F1, plays a different octave
from the instrument.
\vfill\eject

{\title The Edit menu}
\smallskip\hrule height 1.5pt
\bigskip
{\sl Delete Marked Range}
\smallskip
After you have set start and end markers, use this
submenu to delete the range.  There is no way to
recover a deleted area.
\bigskip
{\sl Discard this Sample}
\smallskip
This option removes the currently highlighted sample from
memory.  Deleted samples can not be recovered.
\bigskip
{\sl Insert marked range}
\smallskip
This option will insert part of one sample into another or
even the same sample.  First select the sample that
contains the sound you wish to copy.  Use the end and
start gadgets to select the desired range. Now select
a new sample that you wish to copy the previously marked
range into.  Move the position gadget to the point where
the marked range is to be copied into.  Now pick the
{\sl ``Insert~marked~range''\/} menu.
You will be asked to click the sample where the
marked range is to come from.  Point at the sample
you had previously marked a range in, and click the left
button.  The marked range from the sample just clicked will
be copied into the currently highlighted sample at the
position marked by your
pos'n
gadget.
\bigskip
{\sl Change Playback Period}
\smallskip
Use this submenu to change the period at which
the current sample will be played at. One use
for this is to change the sex of a voice. For example,
a male voice will sound female if it is sped up
(the period is decreased.)
\smallskip
Once you select this option, a requester will appear with
a proportional gadget and a number. You can adjust the
slider gadget to change the number, or click the
number and enter a value directly.  The slider will
have a limited range, but by clicking the number and
typing in the value, any number can be entered.
\smallskip
Note that the Amiga hardware is limited to a minimum playback
rate of 124.
\vfill\eject
{\sl Copy Range to New Slot}
\smallskip
This option creates a new slot with the currently
marked range.  An easy way to edit is to break up
a sample into a lot of little samples -- one word per
slot for example. Then use the
{\sl ``Append~Slot~to~Slot''\/} option to
put them back together in any order.
\bigskip
{\sl Append Slot to Slot}
\smallskip
This option allows you to append any two slots together.
You will be asked to click the slot to append, and
then to click the slot to append it to. Just follow the
prompts.
\bigskip
{\sl Create Stereo}
\smallskip
This option allows you to merge two mono sounds into a stereo sample.
You will be asked to point to and click the slot for the right
and left channels.
\bigskip
{\sl Break up Stereo}
\smallskip
This option create two mono samples out of the two channels
in a stereo sample.
\vfill\eject

{\title Special}
\smallskip\hrule height 1.5pt
\bigskip
{\sl Flip this Sample}
\smallskip
This submenu will reverse the current sample. If you now
play the sample, it will play backwards.  This is useful
for listening to your
{\sl Wings}
records backwards.
\bigskip
{\sl Graph Marked Range}
\smallskip
This will draw a graph of the sound between the start and end
markers. For large samples, the graph will loose realism since
each point plotted represents a large number of actual points.
By moving
the markers, however, you can zoom in on a point to get a truly
accurate graph.  Two horizontal lines above and below the graph
show the minimum and maximum values possible.  If your waveform
hits these lines often, it may indicate distortion due to the input
signals going outside the range of your digitizer.  Lowering the input
gain with knobs on the front of the digitizer will help.  See the
section {\sl ``Setting your digitizer''.}
\smallskip
When you graph a range, the first and last samples will be printed
at the bottom of your screen.  These numbers are the digitized
value of the first and last points of the displayed waveform.
This information is useful for connecting samples ``smoothly.''
Also, these numbers are helpful in creating instruments. The
repeating waveform should start and end with values that are
close to each other.
\bigskip
{\sl Create instrument}
\smallskip
This option is used to create an instrument in IFF format
for use in music editors such as {\sl Instant~Music.}
\smallskip
An instrument file consists of one or more octaves. Each octave
has two sections, a repeating part and a non-repeating
part.  The non-repeating part of a sound is the
first part of the sound, called the attack -- for example, pressing
the key on a piano
causes an inital sound. This is the non-repeating part.
\smallskip
Following the non-repeating part of an octave is an optional
repeating part.  This is sampled data that can be played
over and over to create a continuous tone. Some instruments,
such as a drum, have no repeating part.
Most instruments will have both parts.
\smallskip
The IFF file format requires that the octaves of an instrument
be in order of decreasing frequency. Each successive octave needs to
be twice as long as the previous octave. The repeating portion
of each octave within an instrument should have the same number
of cycles.
Since the next octave is twice as long and has the same number of
cycles, if it is played at the same playback rate the frequency
will be half, thus it will be one octave lower in pitch.
\smallskip
If this is all very confusing, see the file on this disk,
{\sl FORMAT\_IFF\/}. This is technical information on IFF instrument
files as put out by {\sl Electronic~Arts\/} and {\sl Commodore-Amiga\/}.
\smallskip
To create an instrument with
{\sl Perfect Sound},
do the following:
\smallskip
\item{1.} Clear all slots of samples except those which are octaves of
the instrument to be created.
\smallskip
\item{2.} Create each octave in a separate slot. Use the start of range
marker to indicate where the repeat portion of an octave starts.
If there is to be no repeat part, move the start marker to the
far right.
\smallskip
\item{3.} Save your octaves separately -- an instrument can not
easily be broken up into its octaves.
\smallskip
\item{4.} Pick the menu option {\sl "Create Instrument"}
\bigskip
{\sl Create Instrument.}
\smallskip
This will merge your octaves into a new instrument. The octaves
will be arranged in order of decreasing frequency, and you
will be warned if any octaves have a length that is not
twice the previous length.
\smallskip
Note: to play the notes of an instrument, use the function
keys at the top of your keyboard. You can also load an
instrument from
{\sl Deluxe~Video\/}
or
{\sl Instant~Music\/}
and play it with the function keys.
\bigskip
{\sl Freq=Freq*2}
\smallskip
This command creates a new sample with twice the frequency
and half the length of
the current sample. This is done by removing every
other sample from the current sample.
\bigskip
{\sl Freq=Freq/2}
\smallskip
This function takes a sample and divides its frequency by two
without changing the playback period.  This is done by doubling
the sample size and creating new samples by interpolating between
existing samples.

This command is useful for creating instruments. You can digitize
one octave, then create other octaves using the commands
{\sl ``Freq=Freq/2''\/} and {\sl ``Freq=Freq*2.''\/}
\bigskip
{\sl Auto Graph}
\smallskip
Use the submenu to turn {\sl ``auto graphing''} on or off. When
{\sl ``auto graph''} is on, the graph will be updated every time
you change samples or adjust the start or end markers.
\vfill\eject

{\title Digitize}
\smallskip\hrule height 1.5pt
\bigskip
{\sl Alter Record Speed}
\smallskip
This option allows you to enter a new recording
period. Enter the period for the desired playback rate. A larger
period results in poorer sound quality but uses
less memory.  The software can only record at discrete
intervals, so the period you get will not be exactly what you request,
but will be as close as possible rounded down.
See the Hardware manual for a technical discussion
on the exact meaning of the period.
\smallskip
Note: a period of 360 is approximately 10,000 samples per
second.
\bigskip
{\sl View Signal Levels}
\smallskip
This option will display the actual numbers being
received by the analog to digital converter.  You will
see the minimum received so far, the maximum received so far,
and the actual value. These numbers are displayed for
both the left and right channels. Also, the difference
between the two channels is displayed. You can use the
numbers for adjusting the knobs on the front of
your digitizer.  See the section {\sl ``Setting~your~digitizer''\/}
for a complete discussion on how
to adjust your digitizer.
Note that silence is not received as a zero. A zero input
voltage (silence) will digitize to 128 (ideally). A number
smaller than this represents a negative voltage, and
a larger number represents positive voltage.  This is
different than the digital values used by the Amiga's digital
to analog converters.  These converters use zero for
silence, negative numbers for negative voltage, and positive
numbers for positive voltage.  The software
makes the conversion when a sound is recorded.
\bigskip
{\sl Monitor digitizer}
\smallskip
This option will cause {\sl Perfect~Sound\/} to play
the current input sound.
Nothing is being recorded, the sound is just being passed
through from the digitizer to the Amiga sound channel.
Use the submenu to pick
which channel should be used (Right, Left, or Stereo).
\vfill\eject

{\sl Record a Sample}
\smallskip
This option will put you in monitor mode. However, as
soon as you click the left mouse button, you will
start recording. Click the left mouse button again
to halt recording.
Use the submenu to select which channels to use.
\vfill\eject

{\title File}
\smallskip\hrule height 1.5pt
\bigskip
{\sl Load}
\smallskip
Use this to load a sample.  The editor will automatically
recognize IFF
format. If the file format is not recognized, the file will
be loaded as raw data.  If this is the case, no
playback speed will be available and you will have
to set it.
\smallskip
Selecting the
load
menu will bring up the disk I/O requester.  The current
drawer
or directory will be displayed.  Use the slider on the right
side to move around in the directory. To select a directory
or file, just click it.  You can also click the file and
drawer gadgets below the file names in order to type in
a new name directly. To select the internal drive root
directory, enter
{\bf df0:}
and to select the first external drive, enter
{\bf df1:}
in the drawer box.
Once you have selected the desired file and directory,
click the
{\bf load}
box at the bottom of the requester.
\bigskip
{\sl Save As ...}
\smallskip
This option saves the current sample to the current directory.
Type the filename into the disk I/O requester. Use the submenu
to pick which format to use -- IFF, DUMP or COMP.
\smallskip
{\bf IFF} saves the file as a {\sl Form 8SVX\/} in an IFF file,
the standard format for sound on the Amiga. Most programs, such as
{\sl Deluxe~Video\/} and {\sl Instant~Music\/}, expect IFF files.
\smallskip
{\bf DUMP}  will save only digitized sound data with no other
information. Stereo sound will save only one channel.
If you wish to dump stereo sound, break it up into
two samples (see edit menu) and save each sample
separately.
\smallskip
{\bf COMP} saves the file in IFF format using
{\sl Fibonacci~Delta~Compression\/}, a
technique that stores each sample as a 4 bit offset from the previous
sample. This will cut your file size in half, but may reduce the
sound quality considerably.
\bigskip
{\sl Save}
\smallskip
This option saves the current sample using that sample name
as the file name. Pick which format to use with the
submenu.
\vfill\eject

{\title Notes}
\smallskip\hrule height 1.5pt
\vfill\eject
\nopagenumbers
\def\raggedright{\rightskip=0pt plus 3em\spaceskip=.3333em
                 \xspaceskip=.5em}

\raggedright
\end
